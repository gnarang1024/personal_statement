Disaster relief is a fascinating problem domain where many current solutions
are telerobotic systems. However, advancements in sensing, computing, and
manufacturing technology, coupled with the high cost of training specialized
personnel, have caused a recent surge of research interest in developing
autonomous agents. The DARPA robotics challenge is a testament to the shift in
research trends. In graduate school, I would like to investigate the use of
more advanced object recognition techniques, such as scene parsing, in order to
provide planning cues to robots navigating in disaster zones and other dynamic
environments.  These settings contain high degrees of uncertainty and would be
inappropriate to model with static obstacle assumptions. Effective object
recognition in such environments could be invaluable, setting the stage for new
approaches towards planning and decision making. I can see these technologies
having a drastic effect on how we handle natural disasters and crisis, which
would ultimately save more lives. Becoming a professor in the field of robotics
has become my dream. Being a professor is unique in the sense that I could
simultaneously commit myself to developing useful technology that helps and
protects people, and - just as Professor Spear, Professor Spletzer, and my
mentor Dylan had done for me - I could make a difference in the lives of those
I would inevitably teach. I finally discovered a path that is consistent with
my own passions and yet honors a nearly century long family legacy of helping
others.

I am incredibly grateful for the education Lehigh has given me. Having worked
in such a small laboratory, I was forced early on to learn as much as I could
about various disciplines from manufacturing, to controls design, to planning
and perception. In addition, working within more modest research budgets made
me appreciate the tools I have at my disposal. Furthermore, aside from the fact
Georgia Tech has one of the best robotics programs in the country, I want to
earn my PhD at Georgia Tech because of the resources it has to offer. The
greatest resource I believe Georgia Tech has to offer is it's people. After
having heard so much about the DARPA Urban Challenge since I first started
working at VADER, I became increasingly interested in the other teams, which
eventually led me to the research published by talented faculty such as Tucker
Balch, Henrik Christensen, and Tom Collins. Having the chance to work with
people so talented and with such similar research interests would be
invaluable.  However, it is also a matter of numbers. Simply based on the
number of talented people working at the Institute for Robotics one is
constantly exposed by a plethora of ideas and information from a broad range of
fields. The education Georgia Tech has to offer me is only available at less
than a handful of places on the planet. Not only can I understand what a
privilege it would be to a part of the community, but I would use that
privilege to the best of my ability in order to pursue my dream and honor the
institution that would make it possible.
