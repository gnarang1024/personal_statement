\documentclass{article}
\usepackage[left=2.5cm, right=2.5cm, top=3cm, bottom=3cm]{geometry}
\begin{document}
\author{Armon Shariati}
\title{}
\date{}
\maketitle

Forty-five years ago, the Iranian monarchy was dissolved to pave the way for
the Islamic Republic. In the process, members of the generations old Persian
aristocracy were detained, had all their possessions stripped from them, and
often times were executed. However, many of those who rose to power during this
time never forgot how they were treated by the aristocratic families that
governed their lands. Furthermore, by the virtue of the great lengths my
mother's family - one of the oldest and most influential - had gone to protect
and provide for the people they were responsible for, many gave their lives to
ensure they were spared. This is the legacy my mother always sought to uphold.
She had told me that she became a transplant surgeon for three reasons. First,
it had always been her passion since she saw an American surgeon save her
mother's life.  Second, protecting human life had always been a sacred
cornerstone of her family's values. Thirdly, she wanted to ensure that her
children's ambitions would never be driven by money. Ultimately, she was
successful. My mother used her resources to educate her children about the
world.  By my eighteenth birthday, I had studied abroad in four different
countries and visited twelve.  I had volunteered as both an EMT at a Belizean
medical clinic and a skipper's assistant aboard a catamaran touring the
Caribbean.  None of the luxuries or experiences life had to offer were out of
reach. Yet, despite all of these indulgences, I felt empty. I had no passion.
I was terrified by the thought of spending my life being the last link in a
long chain of heroes, and to do nothing to uphold that tradition.

I arrived at Lehigh University insatiably curious. I had applied as an English
major, with a strong interest in Philosophy.  Up until this point, I had
thought literature had been my calling. However, I knew my zeal for English
paled in comparison to how my mother felt about medicine. By some twist of
fortune, I was forced to take Introduction to Computer Science my first
semester because all of the electives I was interested in, such as Greek
Tragedy, were full. I did not expect to see that the same flow of logic I used
to analyze literature appeared in writing software. Quickly, my interest
devolved into an obsession.  Within the first few weeks of the semester I
tirelessly asked numerous professors for opportunities to work with them.  With
absolutely zero background in technology, my pleas fell on deaf ears.
Professor Michael Spear was the only one to take me seriously. Together, we
developed a robotics platform that leveraged the computational capacity and
compact form factor of Android smart phones. We created a fun set of demos and
took them to a local elementary school to generate interest in STEM fields.
Looking back, the task was charmingly elementary. However, while the
satisfaction of instilling the smallest amount of excitement into the
elementary school students was a sufficient reward itself, I learned much in my
time working with Professor Spear.  I learned how to manage frustration, to
become self-reliant, and the importance of education outside the classroom, all
of which became the seeds to my later love for research.

Professor Spear noticed my growing interest in robotics and put me in touch
with Professor John Spletzer. I began working there soon after. It was
horrible...or so I had thought. My mentor's name was Dylan Schweisinger, a PhD
student. Dylan did his best to break down every piece of my self-esteem. He
would give me work far beyond my abilities, berate me for my ignorance, and
ultimately make me question if this was indeed the path I wanted to take. I
knew that this had to be my passion, because in those lowest moments of
self-doubt and confusion, I constantly found the answer to be \emph{yes}. At
the time, I did not appreciate what Dylan had done for me. I had been full of
pride in light of my previous accomplishments. By forcing me to confront my own
ignorance and ego, I became humbled and patient, ready to begin learning for
its own sake. Since I started working in the VADERLAB nearly a year and a half
ago, I have spent every day under Dylan's tutelage.  He taught me how to
engineer software by making me wallow and suffer with any code I wrote. He
taught me how to start generalizing theoretical mathematical concepts for use
as engineering tools. He taught me to take pride in any aspect of development I
am a part of, and to never deliver any less than my best work. In time, Dylan
became one of my best friends. Only then did I understand that he wouldn't have
pushed me as hard as he did if he did not believe I had the potential and
mindset to go the distance.

At the end of my junior year, I was selected to participate in the NSF
sponsored, GRASP REU program at the University of Pennsylvania during the
summer of 2014. I had originally been accepted to the program to work with
Prof. CJ Taylor on vision based mobile systems, which had also been the focus
of my previous work in VADER. To my surprise, upon my arrival, Prof. Taylor
asked me to work on a project where I had no prior experience with either of
the fields involved, graphics and haptics. The vision for the project was to
develop a general platform for creating virtual reality applications that
integrate cutting edge haptic devices with state of the art gaming displays. In
addition to having no prior experience with either discipline, I had also never
been responsible for the entire design of a system of this size from scratch.
Furthermore, I did not even originally believe that I would be interested in
graphics or haptics.  However, using the lessons I had learned from past
research experience, I quickly set up a development environment, acquainted
myself with the basics of graphics and haptics, and devised a plan to have a
working system by the end of the summer. I also discovered how fascinating both
subjects could be. At the end of the program, I had finished a demo and even
won the best paper award. In the fall, I had an abstract accepted by the
Northeast Robotics Colloquium, where I presented my work in a poster session.
However, more important than the successful project itself was learning to stay
open minded when dealing with technologies I am unfamiliar with, and learning
to quickly absorb whatever knowledge is necessary to complete a task.

Finally, at the onset of my senior year, I have finally acquired the skills to
spearhead a graduate-level research project, the project I longed to work on,
our mapping tricycle. The tricycle system begins by taking a series of planar
LIDAR scans. From the aggregated scans, we segment out pole-like landmarks such
as street signs, lamp posts, and parking meters, and assign each landmark a GPS
coordinate on a UTM plane.  The end result is a 2D map of landmarks, which an
autonomous wheelchair can use to negotiate paths between locations of interest
in urban settings. As it passes into my hands, the hardware is mostly complete
alongside the device drivers. My task is to write the necessary software for
robust pose estimation, pole segmentation, and finally, map generation via
SLAM. I am beginning to comprehend what it means to be a researcher. With no
more than a few suggestions in passing from my advisor or graduate mentor as I
incur some of the largest challenges, I am responsible for thinking of my own
creative solutions, independently devising experiments, and assessing their
merit.  Perhaps the most daunting challenge of all is finding the time  and
energy to address the other aspects of my life as I constantly hear myself
saying, \emph{let me just finish this one last thing}. Although, when I come
home at night and lay on my bed exhausted, I can't help but to feel the most
overwhelming sensation of gratification.

In graduate school, I hope to continue working in vision driven robotics.
Specifically, I would like to explore object recognition via scene parsing.
Disaster relief is a fascinating problem domain where currently, many solutions
are telerobotic systems. However, I believe that with new advances in sensing
technology, rising costs of training specialized personnel, and the challenges
of relaying sufficient information to the operators, autonomous agents could
play a larger role. Effective object recognition in such environments would be
invaluable, setting the stage for new approaches towards planning and decision
making. I can see these technologies having a drastic effect on how we handle
natural disasters and crisis, which would ultimately save more lives. Becoming
a professor in the field of robotics has become my dream. Being a professor is
unique in the sense that I could simultaneously commit myself to developing
useful technology that helps and protects people, and - just as Professor
Spear, Professor Spletzer, and Dylan had done for me - I could make a
difference the lives of those I would inevitably teach. I finally discovered a
path that is in line with my own passions and yet honors a legacy I am a part
of, just like my mother.


I am incredibly grateful for the education Lehigh has given me. Having worked
in such a small laboratory, I was forced early on to learn as much as I could
in various disciplines, from manufacturing, to controls design, to planning and
perception. In addition, shallower pockets made me appreciate the tools I have
at my disposal. Furthermore, aside from the fact the University of
Pennsylvania’s GRASP Laboratory has one of the best robotics programs in the
country, I want to earn my PhD at Penn because of the resources it has to
offer. The greatest resource I believe GRASP has to offer is it's people.
Earning my PhD at Penn would mean having the opportunity to work alongside my
superheroes: CJ Taylor, Daniel Lee, and Vijay Kumar. The research these
individuals produce always leads them to becoming a focal point of our lab's
idolatry. After having had the opportunity to work alongside such larger than
life figures during my time as a summer REU student, I knew that these were the
people I would want to continue learning from and growing with because of their
energy, their knowledge, how aligned their research interests were with my own.
However, it is also a matter of numbers. Simply based on the number of talented
people working in GRASP, one is constantly exposed by a plethora of ideas and
information from a broad range of sub fields.  I remember every day, as I would
get a drink at the water fountain,  I would end up engaged in some sort of
conversation, completely outside of the scope of what I would have seen alone
and what I was doing, with some graduate student or professor. The education
Penn has to offer me is only available at less than a handful of places on the
planet. Not only can I understand what a privilege it would be to a part of the
community, but I would use that privilege to the best of my ability in order to
pursue my dream and honor the institution that would make it possible.


\end{document}
