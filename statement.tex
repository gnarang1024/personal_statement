\documentclass{article}
\usepackage[left=2.5cm, right=2.5cm, top=3cm, bottom=3cm]{geometry}
\begin{document}
\author{Armon Shariati}
\title{Personal Statement}
\date{}
\maketitle

Forty-five years ago, the Iranian monarchy was dissolved to pave the way for
the Islamic Republic. In the process, members of the generations old Persian
aristocracy were detained, had all their possessions stripped from them, and
often times executed. However, many of those who rose to power during this time
never forgot how they were treated by the aristocratic families that governed
their lands. Furthermore, by the virtue of the great lengths my mother's family
- one of the oldest and most influential - had gone to protect and provide for
the people they were responsible for, many gave their lives to ensure they were
spared. This is the legacy my mother always sought to uphold. She had told me
that she became a transplant surgeon for three reasons. First, it had always
been her passion since she saw an American surgeon save her mother's life.
Second, protecting human life had always been a sacred cornerstone of her
family's values. Thirdly, she wanted to ensure that her children's ambitions
would never be driven by money. Ultimately, she was successful. My mother used
her resources to educate her children about the world.  By my eighteenth
birthday, I had studied abroad in four different countries and visited twelve.
I had volunteered as both an EMT at a Belizean medical clinic and a skipper's
assistant aboard a catamaran touring the Caribbean.  None of the luxuries or
experiences life had to offer were out of reach. Yet, despite all of these
indulgences, I felt empty. I had no passion.  I was terrified by the thought of
spending my life being the last link in a long chain of heroes, and to do
nothing to uphold that tradition.

I arrived at Lehigh University insatiably curious. I had applied as an English
major, with a strong interest in Philosophy.  Up until this point, I had
thought literature had been my calling. However, I knew my zeal for English
paled in comparison to how my mother felt about medicine. By some twist of
fortune, I was forced to take Introduction to Computer Science my first
semester because all of the electives I was interested in, such as Greek
Tragedy, were full. I did not expect to see that the same flow of logic I used
to analyze literature appeared in writing software. Quickly, my interest
devolved into an obsession.  Within the first few weeks of the semester I
tirelessly asked numerous professors for opportunities to work with them.  With
absolutely zero background in technology, my pleas fell on deaf ears.
Professor Michael Spear was the only one to take me seriously. Together, we
developed a robotics platform that leveraged the computational capacity and
compact form factor of Android smart phones. We created a fun set of demos and
took them to a local elementary school to generate interest in STEM fields.
Looking back, the task was charmingly elementary. However, while the
satisfaction of instilling the smallest amount of excitement into the
elementary school students was a sufficient reward itself, I learned much in my
time working with Professor Spear.  I learned how to manage frustration, to
become self-reliant, and the importance of education outside the classroom, all
of which became the seeds to my later love for research.

Professor Spear noticed my growing interest in robotics and put me in touch
with Professor John Spletzer. I began working there soon after. It was
horrible...or so I had thought. My mentor's name was Dylan Schweisinger, a PhD
student. Dylan did his best to break down every piece of my self-esteem. He
would give me work far beyond my abilities, berate me for my ignorance, and
ultimately make me question if this was indeed the path I wanted to take. I
knew that this had to be my passion, because in those lowest moments of
self-doubt and confusion, I constantly found the answer to be \emph{yes}. At
the time, I did not appreciate what Dylan had done for me. I had been full of
pride in light of my previous accomplishments. By forcing me to confront my own
ignorance and ego, I became humbled and patient, ready to begin learning for
its own sake. Since I started working in the VADERLAB nearly a year and a half
ago, I have spent every day under Dylan's tutelage.  He taught me how to
engineer software by making me wallow and suffer with any code I wrote. He
taught me how to start generalizing theoretical mathematical concepts for use
as engineering tools. He taught me to take pride in any aspect of development I
am a part of, and to never deliver any less than my best work. In time, Dylan
became one of my best friends. Only then did I understand that he wouldn't have
pushed me as hard as he did if he did not believe I had the potential and
mindset to go the distance.

Finally, at the onset of my senior year, I have finally acquired the skills to
spearhead a graduate-level research project, our mapping tricycle. The tricycle
system begins by taking a series of planar LIDAR scans. From the aggregated
scans, we segment out pole-like landmarks such as street signs, lamp posts, and
parking meters, and assign each landmark a GPS coordinate on a UTM plane.  The
end result is a 2D map of landmarks, which an autonomous wheelchair can use to
negotiate paths between locations of interest in urban settings. Honestly, the
hardest part is to find the will to address other aspects of my life as I pour
every ounce of energy into the project. Doing research in robotics is the most
joyful experience I have yet encountered. Becoming a professor in the field of
robotics has become my dream. Insofar I can not only commit my time to
developing some of the most useful technology that helps and protects people,
but lends itself to making a difference in the lives of those you inevitably
teach just as Professor Spear, Professor Spletzer, and Dylan had all done for
me. I finally discovered a path that is in line with my own passions and yet
honors a legacy I am a part of, just like my mother.

Aside from the fact the University of Pennsylvania’s GRASP Laboratory has one
of the best robotics programs in the country, I want to earn my PhD at Penn
because that would mean having the opportunity to work alongside my superheroes
that we always talk about in my lab at Lehigh: CJ Taylor, Max Mintz, and Vijay
Kumar. Following my summer experience working as an REU student, I want the
opportunity to learn from them again. The sensation of working alongside those
as passionate, talented, and driven was unforgettable.  I bring along a strong
grasp of the fundamentals of mobile robotics. Having worked in such a small
laboratory, I was forced early on to learn as much as I could in various
disciplines, from manufacturing, to controls design, to planning and
perception. More importantly, I hope to bring my zest for learning and the
desire to help those around me. 

\end{document}
